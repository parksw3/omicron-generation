\documentclass[12pt]{article}
\usepackage[top=1in,left=1in, right = 1in, footskip=1in]{geometry}

\usepackage{graphicx}
\usepackage{xspace}
%\usepackage{adjustbox}

\newcommand{\comment}{\showcomment}
%% \newcommand{\comment}{\nocomment}

\newcommand{\showcomment}[3]{\textcolor{#1}{\textbf{[#2: }\textsl{#3}\textbf{]}}}
\newcommand{\nocomment}[3]{}

\newcommand{\jd}[1]{\comment{cyan}{JD}{#1}}
\newcommand{\swp}[1]{\comment{magenta}{SWP}{#1}}
\newcommand{\bmb}[1]{\comment{blue}{BMB}{#1}}
\newcommand{\djde}[1]{\comment{red}{DJDE}{#1}}

\newcommand{\eref}[1]{Eq.~(\ref{eq:#1})}
\newcommand{\fref}[1]{Fig.~\ref{fig:#1}}
\newcommand{\Fref}[1]{Fig.~\ref{fig:#1}}
\newcommand{\sref}[1]{Sec.~\ref{#1}}
\newcommand{\frange}[2]{Fig.~\ref{fig:#1}--\ref{fig:#2}}
\newcommand{\tref}[1]{Table~\ref{tab:#1}}
\newcommand{\tlab}[1]{\label{tab:#1}}
\newcommand{\seminar}{SE\mbox{$^m$}I\mbox{$^n$}R}

\usepackage{amsthm}
\usepackage{amsmath}
\usepackage{amssymb}
\usepackage{amsfonts}
\usepackage[utf8]{inputenc} % make sure fancy dashes etc. don't get dropped

\usepackage{lineno}
\linenumbers

\usepackage[pdfencoding=auto, psdextra]{hyperref}

\usepackage{natbib}
\bibliographystyle{unsrt}
\date{\today}

\usepackage{xspace}
\newcommand*{\ie}{i.e.\@\xspace}

\usepackage{color}

\newcommand{\Rx}[1]{\ensuremath{{\mathcal R}_{#1}}\xspace} 
\newcommand{\Ro}{\Rx{0}}
\newcommand{\Rc}{\Rx{\mathrm{c}}}
\newcommand{\Rs}{\Rx{\mathrm{s}}}
\newcommand{\Rres}{\Rx{\mathrm{res}}}
\newcommand{\Rinv}{\Rx{\mathrm{inv}}}
\newcommand{\RR}{\ensuremath{{\mathcal R}}\xspace}
\newcommand{\Rhat}{\ensuremath{{\hat\RR}}}
\newcommand{\Rintrinsic}{\ensuremath{{\mathcal R}_{\textrm{\tiny intrinsic}}}\xspace}
\newcommand{\tsub}[2]{#1_{{\textrm{\tiny #2}}}}
\newcommand{\dd}[1]{\ensuremath{\, \mathrm{d}#1}}
\newcommand{\dtau}{\dd{\tau}}
\newcommand{\dx}{\dd{x}}
\newcommand{\dsigma}{\dd{\sigma}}

\newcommand{\rx}[1]{\ensuremath{{r}_{#1}}\xspace} 
\newcommand{\rres}{\rx{\mathrm{res}}}
\newcommand{\rinv}{\rx{\mathrm{inv}}}

\newcommand{\psymp}{\ensuremath{p}} %% primary symptom time
\newcommand{\ssymp}{\ensuremath{s}} %% secondary symptom time
\newcommand{\pinf}{\ensuremath{\alpha_1}} %% primary infection time
\newcommand{\sinf}{\ensuremath{\alpha_2}} %% secondary infection time

\newcommand{\psize}{{\mathcal P}} %% primary cohort size
\newcommand{\ssize}{{\mathcal S}} %% secondary cohort size

\newcommand{\gtime}{\tau_{\rm g}} %% generation interval
\newcommand{\gdist}{g} %% generation-interval distribution
\newcommand{\idist}{\ell} %% incubation period distribution

\newcommand{\total}{{\mathcal T}} %% total number of serial intervals

\usepackage{lettrine}

\newcommand{\dropcapfont}{\fontfamily{lmss}\bfseries\fontsize{26pt}{28pt}\selectfont}
\newcommand{\dropcap}[1]{\lettrine[lines=2,lraise=0.05,findent=0.1em, nindent=0em]{{\dropcapfont{#1}}}{}}

\begin{document}

\begin{flushleft}{
	\Large
	\textbf\newline{
		Epidemic dynamics (and network structures) affect differences in incubation-period and generation-interval distributions of SARS-CoV-2 variants
	}
}
\bigskip

\section*{Summary}

\begin{itemize}
  \item Neglecting growth rate differences can exaggerate or diminish the observed differences in epidemiological distributions (e.g., incubation period distribution and generation-interval distribution) of SARS-CoV-2 variants
  \item Accounting for growth rate differences yield similar incubation period distribution estimates for Delta and Omicron variants
  \item Despite similaries in the incubation period distributions, we generally estimate shorter mean generation intervals for the Omicron variant (2.7--3.9 days) than for the Delta variant (3.9--4.5 days).
  \item Higher reproduction number of the Omicron variant can lead to a faster depletion of susceptible contacts, esepcially within a household, resulting in shorter realized generation intervals. Simulations suggest that the estimated differences in the mean generation interval are consistent with differences driven by susceptible depletion effects in small networks.
\end{itemize}

\end{flushleft}

\pagebreak

\section{Introduction}

Estimating transmission advantages of new SARS-CoV-2 variants is critical to predicting and controlling the course of the ongoing COVID-19 pandemic.
Transmission advantages of invading variants are typically characterized by the ratios of reproduction numbers, $\Rinv/\Rres$, and the differences in growth rates, $\rinv-\rres$.
These two quantities are linked by the generation-interval distributions of the resident and invading variants.
For example, an invading variant with shorter generation intervals---defined as the time between infection of the infector and the infectee---will exhibit faster epidemic growth ($\rinv > \rres > 0$) even if their reproduction numbers are identical  $\Rinv = \Rres$.

There is currently a lack of generation-interval estimates of SARS-CoV-2 varaints, in part, due to difficulties in observing actual infection events.
Instead, many researchers primarily focus on comparisons of other transmission intervals, such as time between symptom onsets (also referred to as serial intervals) or testing \citep{abbott2022test} of the infector and the infectee.
However, transmission-interval distributions are subject to dynamical biases and may not be an accurate proxy for the generation-interval distribution.
For example, when the epidemic is growing, a cohort of infectors who developed symptoms at the same time are more likely to have been infected recently, shortening their incubation periods---in other words, their incubation periods will be shorter, on average, than their infectees, causing the mean serial interval to be longer than the mean generation interval \citep{park2021forward}.
Therefore, the observed differences (or the lack thereof) in transmission-interval distributions of the invading and resident variants can exaggerate or undermine true differences in their generation-interval distributions, depending on their growth rate differences.

Here, we re-analyze serial-interval data collected by \citep{backer2021omicron}, representing within- and between-household transmissions of the Delta and Omicron varaints from the Netherlands between 13 and 26 December 2021.
The authors of the original article reported shorter serial intervals (3.5 vs 4.1 days) and incubation periods (3.2 vs 4.4 days) for transmission pairs with and without S gene target failures, but they neglected growth rate differences in their inference.
We first discuss epidemiological context in the Netherlands during the study period and provide corrected estimates for the incubation period.
We combine incubation-period estimates with serial-interval data to infer generation-interval distributions of the Delta and Omicron variants.

\section{Methods}

\subsection{Data}

We analyze time series of reported COVID-19 cases (\url{https://data.rivm.nl/covid-19/}) and proportions of SARS-CoV-2 variants detected (\url{https://www.rivm.nl/coronavirus-covid-19/virus/varianten}) from the Netherlands between 29 November 2021 and 30 January 2022.
Both data are publicly available on the National Institute for Public Health and the Environment (RIVM) website.

Serial interval data are taken from the supplementary material of \cite{backer2021omicron}.
The data are aggregated by the length of the serial interval in days and do not include additional individual-level information, such as exposure dates or symptom onset dates.
The data consists of 2529 transmission pairs and are further stratified by the presence of S gene target failure, week of infectors' symptom onset date (week 50 and 51), and the type of transmission (within or between households).
For simplicity, we refer to transmission pairs with and without S gene target failure as Omicron and Delta transmission pairs, respectively.
See original article for details of data collection.

\subsection{Estimating epidemic growth rates}

To estimate the differences in growth rates of the Delta and Omicron variants, we first calculate the number of COVID-19 cases caused by each variant by multiplying the weekly numbers of cases by the proportion of Delta and Omicron variants detected---we use weekly, instead of daily, time series to remove changes in the numbers of cases caused testing patterns.
We then fit a generalized linear model to obtain smooth trajectories for case time series \citep{wood2001mgcv}.
More specifically, we model the logged weekly numbers of cases caused by each variant as a function of time using a shrinkage version of a cubic spline with restricted maximum likelihood.
Finally, we take the derivative of the predicted logged numbers of cases caused by each variant to obtain time-varying growth rate estimates.

To obtain confidence intervals on the estimated time-varying growth rates, we generate 1000 parameter sets using estimated spline coefficients and the corresponding variance-covariance matrices as means and variance-covariance matrices of multivariate normal distributions.
We calculate time-varying growth rates from each parameter set and take 95\% equi-tailed quantiles to generate 95\% confidence limits. 

\subsection{Estimating forward incubation-period distributions from backward incubation-period distributions}

\cite{backer2021omicron} estimated the incubation periods from 513 individuals (258 Omicron and 255 Delta cases), who showed symptom onsets between 1 December 2021 and 2 January 2022.
They used the methods of \cite{backer2020incubation}, which estimates incubation period by inferring probability distributions of time of infection for each individual from their known exposure dates (and then taking the difference between symptom onset time and potential infection times). 
This method further assumes a uniform distribution over exposure intervals.

In practice, incubation periods (and any other epidemiological delays) can be measured in two ways.
The forward incubation periods are measured from a cohort of individuals who were infected at the same time. 
We expect this forward incubation-period distribution $f_I(\tau)$ to remain constant over the course of an epidemic and therefore provide accurate descriptions of incubation periods.
On the other hand, the backward incubation periods, measured from a cohort of individuals, who developed symptoms at the same time, are sensitive to epidemic growth.
In particular, when incidence of infection is growing exponentially, we are more likely to observe shorter backward incubation periods because there will be relatively more individuals who were infected recently.
Specifically, the backward incubation-period distribution $b_I(\tau)$ corresponds to:
\begin{equation}
b_I(\tau) =\frac{ \exp(-r\tau) f_I(\tau)}{\int_0^\infty \exp(-r x) f_I(x)  \dx},
\label{eq:backward}
\end{equation}
where $r$ represents the epidemic growth rate.
The method of \cite{backer2020incubation} effectively estimates the backward incubation-period distribution as it does not take growth rates into account.

To account for growth rate differences between the Delta and Omicron variants, we take the backward incubation-period distributions, estimated by \cite{backer2021omicron}, and calculate the corresponding forward incubation-period distributions by inverting \eref{backward}:
\begin{equation}
f_I(\tau) = \frac{\exp(r\tau) b_I(\tau) }{\int_0^\infty \exp(rx)  b_I(x)\dx},
\label{eq:forward}
\end{equation}
where $b_I(\tau)$ is assumed to follow a Weibull distribution.
To further account for uncertainties in the original parameter estimates, we rely on a sampling scheme, similar to the one we used for the growth rate analysis (in Section 2.2).
First, we approximate the previously-inferred posterior distributions of the shape and scale parameters of the Weibull distribution using a Gamma distribution---we parametrize the Gamma distribution such that (i) its mean matches the median estimates and (ii) the probability that a random variable following the specified Gamma distribution falls between the lower and upper credible limits matches 95\% \citep{park2020reconciling}.
We draw 1000 samples of the shape and scale parameters (for the backward distribution $b_I(\tau)$) from the specified Gamma distributions and estimate the corresponding forward distribution using \eref{forward}.
We take 95\% equi-tailed quantiles to obtain 95\% confidence intervals. 

\subsection{Estimating forward generation-interval distributions from forward serial-interval distributions}

Dynamical biases in the serial-interval distributions are more complex because the serial interval depends on the incubation periods of the infector and the infectee as well as the generation interval between them.
For example, \cite{backer2021omicron} measured the forward serial-interval distributions from cohorts of infectors who developed symptoms during the same week.
In this case, the forward serial interval $\tau_s$ can be expressed in the form \citep{park2021forward}:
\begin{equation}
\tau_s =  - \tau_{i, 1} + \tau_{g} + \tau_{i, 2},
\label{eq:serial}
\end{equation}
where $\tau_{i, 1}$ represents the backward incubation period of the infector (because all infectors developed symptoms at the same time), $\tau_{g|i,1}$ represents the forward generation interval, and $\tau_{i, 2},$ represents the forward incubation period of the infectee.
When incidence of infection is growing exponentially, infectors are more likely to have shorter incubation periods than their infectees on average, $\mathbb E[\tau_{i, 1}] < \mathbb E[\tau_{i, 2}]$, causing the mean forward serial interval to be longer than the mean forward generation interval ($\mathbb E[\tau_s] > \mathbb E[\tau_{g}]$).

We further note that the mean forward generation interval $\mathbb E[\tau_{g}]$ defined in \eref{serial} is also likely shorter than the mean forward generation interval $\mathbb E[\hat{\tau}_{g}]$ that we wish to estimate because infectors with shorter incubation periods (due to exponential growth) also transmit faster.
To illustrate this point, we write $f_{G|I}(\tau_g|\tau_{i,1})$ to represent forward generation-interval distribution conditional on the incubation period.
Then, we can see that the mean forward generation interval $\mathbb E[\tau_{g}]$ defined in \eref{serial} depends on the exponential growth rate:
\begin{align}
\mathbb E[\tau_{g}] &= \int_0^\infty \int_0^\infty x_g f_{G|I}(x_g|x_{i,1}) b_I(x_{i,1}) \dd x_g x_{i, 1}\\
&= \frac{\int_0^\infty \int_0^\infty x_g \exp(-rx_{i,1})  f_{G|I}(x_g|x_{i,1}) f_I(x_{i,1}) \dd x_g x_{i, 1}}{\int_0^\infty  \exp(-r x) f_I(x) \dx}
\end{align}
On the other hand, the mean forward generation interval of our interest does not depend on the exponential growth rate:
\begin{equation}
\mathbb E[\hat{\tau}_{g}] = \int_0^\infty \int_0^\infty x_g f_{G|I}(x_g|x_{i,1}) f_I(x_{i,1}) \dd x_g x_{i, 1}.\\
\label{eq:meangen}
\end{equation}
Therefore, while we generally expect the mean forward generation interval to be shorter than the mean forward serial interval for a growing epidemic, correlation between symptom onset and transmissibility can also have opposite effects.

Here, we model forward incubation-period $f_I(\tau)$ and generation-interval $f_G(\tau)$ distributions with lognormal distributions.
Both distributions are parameterized by log means, $\mu_I$ and $\mu_G$, and log variances, $\sigma_I^2$ and $\sigma_G^2$.
Then, given the correlation coefficient $\rho$, the forward generation-interval distribution conditional conditional on the incubation period $f_{G|I}(\tau|\tau_{i,1})$ has a mean of $\mu_G + \sigma_G \rho (\log(\tau_{i, 1}) - \mu_I)/\sigma_I$ and a log variance of $\sigma_G^2 (1 - \rho^2)$.
Assuming that the incidence of infection is changing exponentially at rate $r$, the forward serial-interval distribution $f_S(\tau)$ for a cohort of infectors who developed symptoms at time $t = 0$ can be calculated by integrating across infection time of the infector $\alpha_1 < 0$ and of the infectee $\alpha_2 > \alpha_1$ \citep{park2021forward}:
\begin{equation}
f_S(\tau) = \frac{1}{\phi} \int_{-\infty}^0\int_{\alpha_1}^\tau \exp(r \alpha_1) f_{G|I}(\alpha_2 - \alpha_1|- \alpha_1) f_I(- \alpha_1) f_I(\tau - \alpha_2) \dd \alpha_2 \dd \alpha_1,
\label{eq:forwardserial}
\end{equation}
where $\phi$ represents the normalization constant to satisfy $\int f_S(x) \dx = 1$;
$-\alpha_1$ corresponds to the incubation period of the infector;
$\alpha_2 - \alpha_1$ corresponds to the generation interval;
and $\tau - \alpha_2$ corresponds to the incubation period of the infectee.

For a given value of $r$, we first estimate the forward incubation-period distribution from the backward distribution, previously estimated by \cite{backer2021omicron}, using \eref{forward}.
We then approximate the forward incubation-period distribution with a lognormal distribution by matching the mean and standard deviation.
Using this incubation-period distribution, we fit \fref{forwardserial} to the observed serial-interval data by minimizing the negative log-likelihood.
We then calculate the mean forward generation interval using \eref{meangen} and approximate the 95\% confidence interval using the Delta method.
We assume $\rho = 0.75$ throughout based on \cite{sender2021unmitigated}---since we do not have individual-level data on infection and symptom onset times, we expect this parameter to be practically unidentifiable.

\subsection{Estimating instantaneous reproduction number}

We use our estimates of the generation-interval distributions to infer instantaneous reproduction numbers $\mathcal R(t)$ of the Delta and Omicron variant, as well as the ratio between two reproduction numbers.
Estimating the instantaneous reproduction number---defined as the average number of secondary infections that a primary case will generate if epidemiological conditions remain constant \citep{fraser2007estimating}---requires the intrinsic generation-interval distribution $g(\tau)$:
\begin{equation}
\mathcal R(t) = \frac{i(t)}{\int_0^\infty i(t-x) g(x) \dx},
\end{equation}
where $i(t)$ represents incidence of infection.
Here, we approximate the intrinsic generation-interval distribution with the forward generation-interval that we estimate for week 50---when epidemic is growing or decaying exponentially, we expect the forward generation-interval to be a good proxy for the intrinsic generation-interval distribution \citep{champredon2015intrinsic, park2020inferring}.
Then, the instantaneous reproduction number is estimated using the renewal equation \citep{fraser2007estimating}:
Incidence of infection is approximated by shifting the smoothed case trajectories by one week to account for reporting delays.
Confidence intervals are calculated by sampling parameters of the smoothed case trajectories as well as the generation-interval distributions from multivariate normal distributions and repeating the analysis 1000 times.

\section{Results}

\fref{epidemic} summarizes epidemiological context in the Netherlands during the study period.
The first Omicron cases was detected in the Netherlands from a sample collected on 19 November 2021 \citep{backer2021omicron}, during which the number of COVID-19 cases were decreasing (\fref{epidemic}A).
As the Omicron variant continued to spread and increase in proportion (\fref{epidemic}B), the number of COVID-19 cases  increased (\fref{epidemic}A).
Multiplying the proportion of each variant with the number of reported COVID-19 cases further allows us to tease apart differences in their epidemiological dynamics (\fref{epidemic}C).
The number of COVID-19 cases caused by the Delta variant continued to decrease throughout the study period with time-varying growth rates decreasing from $r = -0.01/\mathrm{day}$ (95\% CI: $-0.04/\mathrm{day}$--$0.02/\mathrm{day}$) to $r = -0.09/\mathrm{day}$ (95\% CI: $-0.11/\mathrm{day}$--$-0.07/\mathrm{day}$) by the week of January 16, 2022, and increasing back up to $r = -0.04/\mathrm{day}$ (95\% CI: $-0.07/\mathrm{day}$--$-0.01/\mathrm{day}$).
The number of COVID-19 cases caused by the Omicron variant increased rapidly but decelerated over time with time-varying growth rates decreasing from $r=0.18/\mathrm{day}$ (95\% CI: $0.15/\mathrm{day}$--$0.21/\mathrm{day}$) on the week of December 19, 2021, to $r=0.04/\mathrm{day}$ (95\% CI: $0.00/\mathrm{day}$--$0.09/\mathrm{day}$).
Finally, we find that the growth rate difference between the Delta and Omicron variants decreased over time---this finding echoes our previous work, which showed that growth rate differences change over the course of an epidemic even when the ratio of the reproduction numbers remain constant \citep{park2021roles}.

\begin{figure}[!ht]
\includegraphics[width=\textwidth]{figure_epidemic.pdf}
\caption{
\textbf{Epidemic dynamics on the Delta and Omicron variants in the Netherlands.}
(A) Daily numbers of reported COVID-19 cases in the Netherlands (points).
The solid line represents the 7-day moving average.
Data are publicly available on \url{https://data.rivm.nl/covid-19/}.
(B) Proportion of SARS-CoV-2 variants detected from the Netherlands. Data are publicly available on \url{https://www.rivm.nl/coronavirus-covid-19/virus/varianten}.
(C) Weekly numbers of COVID-19 cases caused by the Delta (black points) and Omicron (orange triangles) variants are estimated by multiplying the weekly numbers of cases (A) with the proportion of each variant (B).
Solid lines and shaded areas represent fitted lines and corresponding 95\% confidence intervals using generalized additive model.
(D) Estimated growth rates of the Delta (black) and Omicron variants (orange) and their growth rate differences (purple).
Lines and shaded areas represent medians and corresponding 95\% confidence intervals.
Growth rates are estimated by taking the derivative of the generalized additive model estimates.
\label{fig:epidemic}
}
\end{figure}

For a cohort of individuals who developed symptoms between 1 December 2021 and 2 January 2022, \citep{backer2021omicron} previoulsy found longer mean (backward) incubation period for the Delta variant than for the Omicron variant (\fref{incubation}A).
However, during this period, the number of COVID-19 cases caused by the Delta and Omicron variants showed opposite patterns of epidemic growth/decay (\fref{epidemic}).
Since the number of COVID-19 cases caused by the Delta variant was decreasing, longer incubation periods would have been more likely to be observed.
In contrast, the rapid growth in the number of COVID-19 cases caused by the Omicron variant would have caused shorter incubation periods to be more likely to be observed.
When we account for these growth rate differences and re-estimate the forward incubation periods, we find that both variants have similar incubation period distributions (\fref{incubation}B)---for illustrative purposes, we assume $r=-0.05/\mathrm{day}$ and $r=0.15/\mathrm{day}$ for the Delta and Omicron variants, respectively.
Although the exact estimate of the mean forward incubation periods of both variants are sensitive to the assumed growth/decay rates, we find similar means across a plausible ranges of growth rates with unclear differences between two variants (\fref{incubation}C).
For example, the mean forward incubation period of the Delta variant changes from 3.8 days (95\% CI: 3.5--4.1 days) to 4.4 days (95\% CI: 4.0--4.8 days) as we change the assumed values of $r$ from $-0.1/\mathrm{days}$ to $0/\mathrm{days}$.
The mean forward incubation period of the Omicron variant changes from 3.8 days (95\% CI: 3.3--4.4 days) to 4.5 days (95\% CI: 3.8--5.6 days) as we change the assumed values of $r$ from $0.1/\mathrm{days}$ to $0.2/\mathrm{days}$.

\begin{figure}[!th]
\includegraphics[width=\textwidth]{figure_incubation.pdf}
\caption{
\textbf{Observed and corrected differences in incubation period distributions of Delta and Omicron variants.}
(A) Posterior median estimates of the observed (backward) incubation periods of the Delta (black) and Omicron (orange) variants by \cite{backer2021omicron}.
(B) Forward incubation-period distributions assuming $r=-0.05/\mathrm{day}$ and $r=0.15/\mathrm{day}$ for the Delta (black) and Omicron (orange) variants, respectively.
(C) Corrected estimates of the forward incubation-period distributions for different assumptions about the growth rates of the Delta (black, solid lines) and Omicron variants (orange, dashed lines).
Middle lines represent median estimates, and lower and upper lines represent corresponding 95\% confidence intervals.
\label{fig:incubation}
}
\end{figure}

Our corrected estimates of the forward incubation period distributions further allow us to infer the forward generation-interval distributions.
For illustrative purposes, we first focus on within-household serial intervals from infectors who developed symptoms during week 50 (13-–19 December, 2021).
\citep{backer2021omicron} previously reported shorter mean serial interval of the Omicron variant (3.5 days) than that of the Delta variant (4.5 days) for this cohort, but the overall shapes of the distributions are similar (\fref{serial}A).
However, when we account for growth rate differences (assuming $r=-0.05/\mathrm{day}$ and $r=0.15/\mathrm{day}$ for the Delta and Omicron variants, respectively), the estimated mean forward generation interval exhibits a larger difference (\fref{serial}B): 3.3 days (95\% CI: 2.8--3.7 days) for the Omicron variant and 4.2 days (95\% CI: 4.0--4.4 days).
Notably, both variants have similar mode, but the transmissibility of the omicron variant decays much faster (\fref{serial}B).

While we estimate shorter mean forward generation intervals for the Omicron variant across all stratifications (\fref{serial}C), the differences in the mean generation intervals are unclear for between-household pairs.
Consistent with the original study, which also repoted shorter mean forward serial intervals for between-household pairs \citep{backer2021omicron}, we estimate shorter mean forward generation intervals for between-household pairs.
We also estimate a reduction in the mean forward generation intervals from week 50 to week 51. 

\begin{figure}[!th]
\includegraphics[width=\textwidth]{figure_compare.pdf}
\caption{
\textbf{Estimated relationship between forward incubation and generation-interval distributions of Delta and Omicron variants.}
(A) Observed and fitted forward serial-interval distributions fpr within-household transmission pairs in the Netherlands for the Delta (black) and Omicron (orange) variants \citep{backer2021omicron}.
Serial intervals are calculated for infectors who developed symptoms on week 50 (13 and 19 December, 2021).
Points represent the observed data.
Lines represent the fitted lines assuming $r=-0.05/\textrm{day}$ for the Delta variant and $r=0.15/\textrm{day}$ for the Omicron variant. 
Bivariate lognormal distributions with correlation coefficient of 0.75 are used to model relationships between incubation and generation intervals.
(B) Estimated forward generation-interval distributions for within-household transmission pairs in the Netherlands.
(C) Sensitivity of the mean generation-interval estimates to assumed growth rates of the Delta (black, solid lines) and Omicron variants (orange, dashed lines) stratified by the week of infectors' symptom onset and the type of transmission pairs.
\label{fig:serial}
}
\end{figure}

Accounting for differences in the generation-interval distributions, we estimate that the reproduction number of the Omicron variant decreased from 1.83 (95\% CI: 1.65--2.05) to 1.17 (95\% CI: 1.00--1.36) between December 12, 2021, and January 23, 2022(\fref{reproduction}A).
On the other hand, the reproduction number of the Delta variant decreased from 0.89 (95\% CI: 0.82--0.97) to 0.68 (95\% CI: 0.62--0.74) between December 5, 2021, and January 9, 2022, and increased back up to 0.82 (95\% CI: 0.72--0.93) by January 23, 2022 (\fref{reproduction}A).
We estimate the reproduction number ratios stayed rouhgly constant at around 2.2 (95\% CI: 2.0--2.5) between December 12--26, 2021, and slowly decreased to 1.4 (95\% CI: 1.2--1.7).
However, if we neglect differences in the generation-interval distributions and solely rely on the generation-interval-distribution estimate for the Delta variant, we over-estimate the reproduction number of the Omicron variant and therefore is transmission advantage (\fref{reproduction}B).
In this case, the reproduction ratio decreases from 2.5 (95\% CI: 2.2--2.8) to 1.5 (95\% CI: 1.2--1.8), corresponding to roughly 7--13\% bias.

\begin{figure}[!th]
\includegraphics[width=\textwidth]{figure_reproduction_advantage.pdf}
\caption{
\textbf{Estimated time-varying reproduction number advantages of the Omicron variant.}
(A) Estimated instantaneous reproduction numbers and their ratios over time while accounting for differences in the generation-interval distributions.
(B) Estimated instantaneous reproduction numbers and their ratios over time while assuming identical generation-interval distributions.
(C--D) Phase diagrams of the reproduction ratio against instantaneous reproduction numbers of the Delta (C) and Omicron (D) variants.
The instantaneous reproduction number of each variant is estimated using the renewal equation by shifting the smoothed case curves by one week (\fref{epidemic}C).
The intrinsic generation-interval distribution is approximated by the maximum likelihood estimates of the generation-interval distributions for within-household transmission pairs based on $r=-0.05$ for the Delta variant (black) and $r=0.15$ for the Omicron variant (orange).
Purple lines represent the ratio between the effective reproduction numbers of the Delta and Omicron variants.
Lines and shaded regions represent medians and corresponding 95\% confidence intervals.
\label{fig:reproduction}
}
\end{figure}

Comparing phase diagrams of the reproduction ratio against instantaneous reproduction numbers of the Delta (\fref{reproduction}C) and Omicron (\fref{reproduction}D) variants helps us better understand the reduction in transmission advantage over time.
As the reproduction number of the Delta variant decreases, the reproduction number ratio initially increases and stays constant. 
Then, the reproduction number ratio and the reproduction number of the Delta variant both decreases at the same time, and the reproduction number of the Delta variant increases slightly near the end (\fref{reproduction}C).
Similarly, the decrease in the reproduction number ratio coincides with the decrease in the reproduction number of the Omicron variant, implying that epidemiological changes driving the dynamic had larger effects on the transmission of the Omicron variant than on the transmission of Delta variant (\fref{reproduction}D);
larger reduction in the reproduction number of the Omicron variant also caused its growth rate to decrease faster, causing changes in the growth rate difference (\fref{epidemic}D).

\section{Discussion}

We compare estimates of the forward incubation-period and generation-interval distributions of the Delta and Omicron variants from the Netherlands.
The original analysis detailing the data set previously reported shorter mean incubation period and serial interval for the Omicron variant \citep{backer2021omicron}.
Accounting for differences in epidemic growth rates, we find similar incubation-period distributions for both variants but a considerably shorter (0.5--0.9 days) mean generation interval for the Omicron variant.
The mean forward generation interval of both variants further decreased between week 50 (13--19 December, 2021) and week 51 (20--26 December, 2021).
Finally, we estimate that the transmission advantage of the Omicron variant decreased from 2.2-fold to 1.4-fold between early December and late January. 
Neglecting differences in the generation-interval distributions can result in $\approx 10\%$ bias in estimates of the transmission advantage.

The generation-interval distribution describes changes in the individual-level transmission dynamics over the course of infection and therefore provides cruicial information for epidemic control.


\pagebreak

\section*{Supplementary Materials}

\begin{figure}[!th]
\includegraphics[width=\textwidth]{figure_compare_he.pdf}
\caption{
\textbf{Estimated relationship between forward incubation and generation-interval distributions of Delta and Omicron variants.}
We model relationships between incubation and generation intervals using a shifted gamma distribution as suggested by \cite{he2020temporal} as a sensitivity analysis.
We obtain similar mean generation-interval estimates, but the predited bivariate relationships (C--D) are considerably different.
This model also gives overall worse fits to the observed serial-interval distribution.
See Figure 3 in the main text for details.
}
\end{figure}

\bibliography{omicron.bib}

\end{document}
