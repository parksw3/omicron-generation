\documentclass[12pt]{article}
%DIF LATEXDIFF DIFFERENCE FILE
%DIF DEL response.tex.HEAD.oldfile   Mon Nov 28 04:52:20 2022
%DIF ADD response.tex                Mon Dec 26 22:43:12 2022
\usepackage[utf8]{inputenc}

\usepackage{color}

\usepackage{xspace}

\usepackage{lmodern}
\usepackage{amssymb,amsmath}

\usepackage[pdfencoding=auto, psdextra]{hyperref}

\usepackage{natbib}
\bibliographystyle{chicago}

\newcommand{\eref}[1]{Eq.~(\ref{eq:#1})}
\newcommand{\fref}[1]{Fig.~\ref{fig:#1}}

\newcommand{\rR}{\mbox{$r$--$\cal R$}}
\newcommand{\RR}{\ensuremath{{\cal R}}}
\newcommand{\RRhat}{\ensuremath{{\hat \cal R}}}
\newcommand{\Rx}[1]{\ensuremath{{\cal R}_{#1}}} 
\newcommand{\Ro}{\ensuremath{{\mathcal R}_{0}}\xspace}
\newcommand{\Rs}{\Rx{\mathrm{s}}}
\newcommand{\Rpool}{\ensuremath{{\mathcal R}_{\textrm{\tiny{pool}}}}\xspace}
\newcommand{\Reff}{\Rx{\mathit{eff}}}
\newcommand{\Tc}{\ensuremath{C}}

\newcommand{\dd}[1]{\ensuremath{\, \mathrm{d}#1}}
\newcommand{\dtau}{\dd{\tau}}
\newcommand{\dx}{\dd{x}}
\newcommand{\dsigma}{\dd{\sigma}}

\newcommand{\rev}{\subsection*}
\newcommand{\revtext}{\textsf}
\setlength{\parskip}{\baselineskip}
\setlength{\parindent}{0em}

\newcommand{\comment}[3]{\textcolor{#1}{\textbf{[#2: }\textsl{#3}\textbf{]}}}
\newcommand{\jd}[1]{\comment{cyan}{JD}{#1}}
\newcommand{\swp}[1]{\comment{magenta}{SWP}{#1}}
\newcommand{\dc}[1]{\comment{blue}{DC}{#1}}
\newcommand{\jsw}[1]{\comment{green}{JSW}{#1}}
\newcommand{\hotcomment}[1]{\comment{red}{HOT}{#1}}

\newcommand{\psymp}{\ensuremath{p}} %% primary symptom time
\newcommand{\ssymp}{\ensuremath{s}} %% secondary symptom time
\newcommand{\pinf}{\ensuremath{\alpha_1}} %% primary infection time
\newcommand{\sinf}{\ensuremath{\alpha_2}} %% secondary infection time

\newcommand{\psize}{{\mathcal P}} %% primary cohort size
\newcommand{\ssize}{{\mathcal S}} %% secondary cohort size

\newcommand{\gtime}{\tau_{\rm g}} %% generation interval
\newcommand{\gdist}{g} %% generation-interval distribution
\newcommand{\idist}{\ell} %% incubation period distribution

\newcommand{\total}{{\mathcal T}} %% total number of serial intervals
%DIF PREAMBLE EXTENSION ADDED BY LATEXDIFF
%DIF UNDERLINE PREAMBLE %DIF PREAMBLE
\RequirePackage[normalem]{ulem} %DIF PREAMBLE
\RequirePackage{color}\definecolor{RED}{rgb}{1,0,0}\definecolor{BLUE}{rgb}{0,0,1} %DIF PREAMBLE
\providecommand{\DIFaddtex}[1]{{\protect\color{blue}\uwave{#1}}} %DIF PREAMBLE
\providecommand{\DIFdeltex}[1]{{\protect\color{red}\sout{#1}}}                      %DIF PREAMBLE
%DIF SAFE PREAMBLE %DIF PREAMBLE
\providecommand{\DIFaddbegin}{} %DIF PREAMBLE
\providecommand{\DIFaddend}{} %DIF PREAMBLE
\providecommand{\DIFdelbegin}{} %DIF PREAMBLE
\providecommand{\DIFdelend}{} %DIF PREAMBLE
\providecommand{\DIFmodbegin}{} %DIF PREAMBLE
\providecommand{\DIFmodend}{} %DIF PREAMBLE
%DIF FLOATSAFE PREAMBLE %DIF PREAMBLE
\providecommand{\DIFaddFL}[1]{\DIFadd{#1}} %DIF PREAMBLE
\providecommand{\DIFdelFL}[1]{\DIFdel{#1}} %DIF PREAMBLE
\providecommand{\DIFaddbeginFL}{} %DIF PREAMBLE
\providecommand{\DIFaddendFL}{} %DIF PREAMBLE
\providecommand{\DIFdelbeginFL}{} %DIF PREAMBLE
\providecommand{\DIFdelendFL}{} %DIF PREAMBLE
%DIF HYPERREF PREAMBLE %DIF PREAMBLE
\providecommand{\DIFadd}[1]{\texorpdfstring{\DIFaddtex{#1}}{#1}} %DIF PREAMBLE
\providecommand{\DIFdel}[1]{\texorpdfstring{\DIFdeltex{#1}}{}} %DIF PREAMBLE
%DIF LISTINGS PREAMBLE %DIF PREAMBLE
\RequirePackage{listings} %DIF PREAMBLE
\RequirePackage{color} %DIF PREAMBLE
\lstdefinelanguage{DIFcode}{ %DIF PREAMBLE
%DIF DIFCODE_UNDERLINE %DIF PREAMBLE
  moredelim=[il][\color{red}\sout]{\%DIF\ <\ }, %DIF PREAMBLE
  moredelim=[il][\color{blue}\uwave]{\%DIF\ >\ } %DIF PREAMBLE
} %DIF PREAMBLE
\lstdefinestyle{DIFverbatimstyle}{ %DIF PREAMBLE
	language=DIFcode, %DIF PREAMBLE
	basicstyle=\ttfamily, %DIF PREAMBLE
	columns=fullflexible, %DIF PREAMBLE
	keepspaces=true %DIF PREAMBLE
} %DIF PREAMBLE
\lstnewenvironment{DIFverbatim}{\lstset{style=DIFverbatimstyle}}{} %DIF PREAMBLE
\lstnewenvironment{DIFverbatim*}{\lstset{style=DIFverbatimstyle,showspaces=true}}{} %DIF PREAMBLE
%DIF END PREAMBLE EXTENSION ADDED BY LATEXDIFF

\begin{document}

\noindent Dear Editor:

Thank you for the chance to revise and resubmit our manuscript \DIFaddbegin \DIFadd{(previously 2022-12304)}\DIFaddend . 
We now discuss potential mechanisms driving shorter generation intervals of the Omicron variant in detail.
In doing so, we distinguish intrinsic generation intervals, which represent changes in infectiousness over the course of infection, from realized generation intervals, which represent time between actual transmission events---our estimates correspond to the latter quantity.
We also discuss the uncertainties associated with possible effects of immunity on the generation-interval distributions of the Omicron and Delta variants.
As the reviewer pointed out, it is true that immune evasiveness of the Omicron variant is the main explanation for the population-level success of Omicron---however, \DIFdelbegin \DIFdel{its success or immune evasion }\DIFdelend \DIFaddbegin \DIFadd{this mechanism }\DIFaddend does not require or depend on changes in generation-interval distributions as we explain. 
\DIFaddbegin 

\DIFaddend Below please find our detailed responses to reviewers.

\rev{Editor}

\revtext{Dear Colleagues:}

\revtext{Both reviewers, as well as I, like this report and the way it is written. Reviewer 2 recommended its acceptance. Reviewer 1 raised an issue you need to address, "WHY the omicron variant had such a short generation time in this setting...", and 
recommend the rejection of your report. I agree with this review about this issue.}

\revtext{Based on that reviewer's criticism, I am recommending what I call a soft rejection. Address this reviewer's criticism in detail and modify the manuscript accordingly. By "address," I am considering the possibility of your presenting a compelling argument for why this issue need not be considered in this report. Your response to the reviews and the revised manuscript will be sent to the second reviewer.}

Thank you very much. We have revised the manuscript to discuss the effect of immunity. In doing so, we have tried to distinguish its population-level effect (which allowed the Omicron variant to dominate) from the individual-level effect (changes in the generation-interval distribution).
We note these two effects are separate: its success or immune evasion does not require or depend on changes in generation-interval distributions.

\rev{Reviewer \#1}

\revtext{The paper commences with a very clear introduction to the problems in estimating the transmission advantage of a new variant - in particular, the role of dynamical bias. This is followed by a very reasonable proposal for how the problem may be addressed. Applying these methods to a re-analysis of data on within- and between- household spread of delta and omicron variants of SARS-CoV-2 suggests that while both had similar incubation-period distributions, the mean generation interval for omicron was shorter by 0.3-0.8 days than previously estimated (0.6 days). The original analysis had also reported a difference in mean incubation periods of 3.2 days for omicron versus 4.4 days for delta; the methods used in this paper suggest that both have a mean of 4.1 days. The consequences of discriminating between generation time intervals on the estimation of transmission advantage is shown in Fig 5, although it is not clear how this pertains to dynamical bias as both the original study and this one reported a difference between the variants in this respect.}

Dynamical biases pertain to the estimation of delay distributions, and not to the estimation of transmission advantage. 
We agree that both the original study and this one report that the Omicron variant has a higher reproduction number. 
Here, what we show is that estimates of transmission advantage depend on assumptions about the generation-interval distributions. 
For example, as we explain in the Discussion, assuming longer generation-interval distributions can lead to further biases in the estimates of transmission advantange:

``For example, \cite{pearson2021bounding} estimated a much higher reproduction advantage of the Omicron variant (\textgreater\ 4-fold) compared to the Delta variant in South Africa but also assumed a longer mean generation interval for the Delta and Omicron variants (6.4 vs 5.2 days, respectively).
With our generation-interval estimates, we estimate a 2.6-fold reproduction advantage for the Omicron variant assuming $r=-0.06$ and $r=0.26$ for the Delta and Omicron variants, respectively---these growth rates were chosen to match the assumptions and results of \cite{pearson2021bounding} for Gauteng province \DIFaddbegin \DIFadd{in South Africa}\DIFaddend .''

\revtext{The caveats of this study, and the problems of applying this method to other datasets, are very thoroughly outlined in the discussion.} 

Thank you.

\revtext{Less attention is paid to the more important question of WHY the omicron variant had such a short generation time in this setting.}

We have now added text to the discussion explaining different types of generation intervals (intrinsic vs realized) and why the realized generation interval of the Omicron variant might be shorter:

``Some studies have also estimated that the Omicron variant has shorter transmission intervals than the Delta variant \citep{abbott2022test,kremer2022observed,song2022serial}, but there has been a lack of direct generation-interval estimates.
\cite{ito2022estimating,selby2022generation} estimated the generation-interval distributions of the Omicron variant but they both relied on population-level epidemic dynamics (rather than individual-level transmission data).
Here, we \DIFdelbegin \DIFdel{we }\DIFdelend estimate a shorter mean \emph{realized} generation interval for the Omicron variant.
The realized generation intervals represent time between actual infection events, and are different from the intrinsic generation intervals, which reflect the average profile of infectiousness of infected individuals over the course of their infections \citep{champredon2015intrinsic}.

Shorter realized generation intervals of the Omicron variant may be driven, in part, by shorter intrinsic generation intervals.
For example, faster within-host clearance of the Omicron variant \citep{hay2022viral} may lead to faster recovery, which could in turn shorten the intrinsic generation interval \citep{roberts2007model}.
Pre-existing immunity has also typically been associated with faster recovery \citep{kissler2021viral} (and therefore, shorter generation intervals).
In our analysis, a greater proportion of Omicron than Delta infections would have been in people with pre-existing immunity (because these people are less likely to be infected by Delta);
this allowed the Omicron variant to rapidly replace the previously dominant Delta variant at the population level.
However, the individual-level relationship between immunity and the intrinsic generation-interval distributions of the Omicron variant is difficult to predict given its high immune evasiveness:
some individuals may have been able to elicit a strong immune response, thereby recovering faster, but many other individuals likely experienced a full course of infection.
In other words, pre-existing immunity may shorten intrinsic generation intervals, whereas immune \DIFdelbegin \DIFdel{evasiveness may lengthen them---their }\DIFdelend \DIFaddbegin \DIFadd{evasion may lengthen them (relative to reinfection with Delta, which would be less immune evasive).
Their }\DIFaddend combined effects on the resulting generation-interval distribution are unclear.
We note that the population-level effects of immune evasiveness, which allowed the Omicron variant to invade rapidly, \DIFdelbegin \DIFdel{are separate from }\DIFdelend \DIFaddbegin \DIFadd{do not depend on these }\DIFaddend possible changes in the generation-interval distribution\DIFdelbegin \DIFdel{that we have outlined above.
Even if Delta and Omicron variants had identical intrinsic generation-interval distributions, immune evasiveness alone would allow the Omicron variant to become dominant.
Therefore, we might expect the differences in the }\DIFdelend \DIFaddbegin \DIFadd{:
immune evasion could have allowed Omicron to replace Delta even without differences in }\DIFaddend intrinsic generation-interval distributions\DIFdelbegin \DIFdel{of the Delta and Omicron infections to be more strongly influenced by the intrinsic differences in their viral kinetics}\DIFdelend .

Even if the Delta and Omicron variants had identical intrinsic generation-interval distributions \DIFaddbegin \DIFadd{(e.g., a lack of changes in viral kinetics or the effect of pre-existing immunity)}\DIFaddend , their realized generation-interval distributions could still differ.
For example, if there are more stringent contact tracing measures against the Omicron variant, individuals who are infected with the Omicron variant would transmit for a shorter amount of time, thereby resulting in a shorter realized generation interval.
Although this was the case in the Netherlands \citep{backer2021omicron}, targeted control measures against a specific variant are likely to have \DIFaddbegin \DIFadd{a }\DIFaddend small impact on the overall generation-interval distribution given that variants are usually identified for a small fraction of total infections.
If Delta and Omicron variants have different degrees of symptomaticity or different incubation periods, behavioral changes after symptom onset, such as self-isolation, could also lead to shorter realized generation intervals.
However, this explanation is also less likely given similarities in the inferred incubation-period distributions.

Instead, we tentatively hypothesize that the differences in realized generation intervals may be primarily driven by the network effect \citep{park2020inferring,hart2022generation}: a higher reproduction number of the Omicron variant leads to faster susceptible depletion among close contacts, which in turn prevents long generation intervals from being realized. 
To illustrate this effect, consider a scenario in which an individual infected with variant A can infect one person per day for four days whereas another individual infected with variant B can infect two people per day for four days---in this case, both variants have identical intrinsic generation-interval distributions.
We note that this scenario does not match our model, which assumes time-varying force of infection, not a deterministic number of infections; nonetheless, this scenario provides a \DIFdelbegin \DIFdel{simplest }\DIFdelend \DIFaddbegin \DIFadd{simple }\DIFaddend example for explaining the network effect\DIFdelbegin \DIFdel{without requiring new simulations or an extensive review of existing literature}\DIFdelend .
Under the same scenario, if each individual \DIFdelbegin \DIFdel{were living in a household containing four other people}\DIFdelend \DIFaddbegin \DIFadd{closely interacted with four other contacts}\DIFaddend , the individual infected with variant A will infect one \DIFdelbegin \DIFdel{household member }\DIFdelend \DIFaddbegin \DIFadd{person }\DIFaddend every day; in contrast, an individual infected with variant B will infect two people per day and no longer transmit (effectively) after the first two days, \DIFdelbegin \DIFdel{thereby having }\DIFdelend \DIFaddbegin \DIFadd{leading to }\DIFaddend shorter realized generation intervals.
Previous simulations showed that such network effects can have considerable impact on realized generation intervals \citep{park2020inferring}.
While the network effect is expected to be strongest among household contacts, it is also applicable to other forms of contact structures that involve repeated contacts between the same group of individuals (because only the first infectious contact results in infection).

Our study indicates that the Omicron variant has a shorter mean realized generation interval than that of the Delta variant, but the difference between the intrinsic infectiousness profiles remains unclear.
In particular, similarities in the incubation-period distributions of the Delta and Omicron variants suggest that the differences in their \DIFdelbegin \DIFdel{true infectiousness profile }\DIFdelend \DIFaddbegin \DIFadd{infectiousness profiles }\DIFaddend may be smaller than the estimated differences in their realized generation-interval distributions.
In addition, the \DIFaddbegin \DIFadd{counter-factual initial }\DIFaddend intrinsic generation intervals \DIFdelbegin \DIFdel{of both Omicron and Delta }\DIFdelend \DIFaddbegin \DIFadd{(for an immunologically naive population) of both }\DIFaddend variants are likely longer than what we estimate given existing levels of interventions, including vaccination, and pandemic awareness---estimating \DIFaddbegin \DIFadd{initial }\DIFaddend intrinsic generation-interval distributions of SARS-CoV-2 variants is expected to be a difficult problem as it requires data from times when awareness levels were low \citep{sender2021unmitigated}.
Nonetheless, estimates of realized generation-interval distributions describe current epidemic dynamics, implicitly accounting for intervention and behavioral effects,
and can therefore be expected to improve estimates of effective reproduction numbers.''

\revtext{The authors discuss the possibility of a "network effect" which is predicated on omicron having a higher Ro and therefore somewhat circular.}

We don't believe that this explanation is circular, and have tried to clarify our explication further. As we explain in the Discussion, the success of the Omicron variant at the population level is due to immune evasion (also pointed out by reviwer 1), and this doesn't require shorter generation \DIFdelbegin \DIFdel{interval}\DIFdelend \DIFaddbegin \DIFadd{intervals}\DIFaddend . We now provide a concrete example for the network effect, explaining how higher reproduction number can lead to shorter \emph{realized} generation intervals even when two variants have identical \emph{intrinsic} generation intervals:

``Instead, we tentatively hypothesize that the differences in realized generation intervals may be primarily driven by the network effect \citep{park2020inferring,hart2022generation}: a higher reproduction number of the Omicron variant leads to faster susceptible depletion among close contacts, which in turn prevents long generation intervals from being realized. 
To illustrate this effect, consider a scenario in which an individual infected with variant A can infect one person per day for four days whereas another individual infected with variant B can infect two people per day for four days---in this case, both variants have identical intrinsic generation-interval distributions.
We note that this scenario does not match our model, which assumes time-varying force of infection, not a deterministic number of infections; nonetheless, this scenario provides a \DIFdelbegin \DIFdel{simplest }\DIFdelend \DIFaddbegin \DIFadd{simple }\DIFaddend example for explaining the network effect without requiring new simulations or an extensive review of existing literature.
Under the same scenario, if each individual were living in a household containing four other people, the individual infected with variant A will infect one household member every day; in contrast, an individual infected with variant B will infect two people per day and no longer transmit (effectively) after the first two days, \DIFdelbegin \DIFdel{thereby having }\DIFdelend \DIFaddbegin \DIFadd{leading to }\DIFaddend shorter realized generation intervals.
Previous simulations showed that such network effects can have considerable impact on realized generation intervals \citep{park2020inferring}.
While the network effect is expected to be strongest among household contacts, it is also applicable to other forms of contact structures that involve repeated contacts between the same group of individuals (because only the first infectious contact results in infection).''

\revtext{Other factors "such as more stringent contact tracing measures against the Omicron variant in the Netherlands [4], faster within-host clearance of the Omicron variant [35], and viral kinetics of reinfection and breakthrough infections" are mentioned without sufficient explanation as to why they would yield such a result.}

We now provide a more detailed explanation.

\revtext{The role of immune evasion - although mentioned in passing - has not been sufficiently accommodated either in the discussion or in the analysis itself, even though it presents itself as the most likely explanation for the success of omicron in almost every global setting.}

We now discuss the possible effects of immune evasion more thorougly. 

\revtext{To address this properly, the authors will have to account for the different proportions of the population currently immune to infection by either variant and how the level of cross-immunity between them affects our observations of transmission advantage. The authors should at least justify why they have not taken such an obvious feature of this multi-strain system into account.}

While we agree that immune evasion is the most likely explanation for the success of omicron in almost every global setting, its success or immune evasion does not require or depend on changes in generation-interval distributions.
We now provide a more thorough explanation for how pre-existing immunity and immune evasiveness of the Omicron variant might shape the generation-interval distribution: 

``Pre-existing immunity has also typically been associated with faster recovery \citep{kissler2021viral} (and therefore, shorter generation intervals).
In our analysis, a greater proportion of Omicron than Delta infections would have been in people with pre-existing immunity (because these people are less likely to be infected by Delta);
this allowed the Omicron variant to rapidly replace the previously dominant Delta variant at the population level.
However, the individual-level relationship between immunity and the intrinsic generation-interval distributions of the Omicron variant is difficult to predict given its high immune evasiveness:
some individuals may have been able to elicit a strong immune response, thereby recovering faster, but many other individuals likely experienced a full course of infection.
In other words, pre-existing immunity may shorten intrinsic generation intervals, whereas immune \DIFdelbegin \DIFdel{evasiveness }\DIFdelend \DIFaddbegin \DIFadd{evasion }\DIFaddend may lengthen them---their combined effects on the resulting generation-interval distribution are unclear.
We note that the population-level effects of immune evasiveness, which allowed the Omicron variant to invade rapidly, \DIFdelbegin \DIFdel{are separate from }\DIFdelend \DIFaddbegin \DIFadd{do not depend on these }\DIFaddend possible changes in the generation-interval distribution\DIFdelbegin \DIFdel{that we have outlined above.
Even if Delta and Omicron variants had identical intrinsic generation-interval distributions, immune evasiveness alone would allow the Omicron variant to become dominant.
Therefore, we might expect the differences in the }\DIFdelend \DIFaddbegin \DIFadd{:
immune evasion could have allowed Omicron to replace Delta even without differences in }\DIFaddend intrinsic generation-interval distributions\DIFdelbegin \DIFdel{of the Delta and Omicron infections to be more strongly influenced by the intrinsic differences in their viral kinetics}\DIFdelend .''

\rev{Reviewer \#2}

\revtext{The significance statement is accurate. I find the paper definitely relevant for future epidemiological studies on COVID-19, and even important for some technical works. But neither the methods nor the results are likely to have a huge impact on the field, even less on other fields.}

\revtext{The manuscript is technically sound and very well written. The results are convincing and clearly explained in the manuscript, and the arguments presented make sense from an epidemiological perspective. This paper emphasizes once more the importance of accounting for the epidemic dynamic (or the differential dynamic between two variants, as in this case) when inferring fundamental epidemiologically relevant quantities that vary in time during the infection. I find it an important contribution to the literature, albeit a technical one.}

Thank you for your review.

\bibliography{omicron}

\end{document}
